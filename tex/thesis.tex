\documentclass[english,version-2020-11]{uzl-thesis}


% Copy this file as a template for your thesis. You will have to take
% action at all places marked by
%
% !!!!!!!!!!!!!!!!!!!!!!!!!!!!!!!!!!
% !!! Your action is needed here !!!
% !!!!!!!!!!!!!!!!!!!!!!!!!!!!!!!!!!
%
% The first place your action is needed is the first line of this
% document:
%
%
% Language of the thesis:
%
% You must use either 'german' or 'english' above, depending on the
% language used in the main text. This will automatically setup a lot
% of things in the background.
%
%
% Version of the class:
%
% You must specify which version of the thesis class is to be
% used. This is important in case the class style changes in later
% years, but we still want an older thesis to look the same, even when
% things are changed in the class.
%
% Do not change or remove the version-xxxx key.
%
%
% Text encoding:
%
% Your thesis *must* be encoded in utf8 (unicode), which is the
% default in most editors these days. Do *not* change this to latin8.



%%%
%
% Main setup:
%
%%%
%
% You must use the \UzLThesisSetup command to specify numerous things
% about your thesis. This includes the entries on the title page, the 
% abstracts, and the bibliography style. You do so by specifying
% so-called "values" for so-called "keys". For instance, 
% for the key "Autor" you must provide your name as the value. You do
% so by writing 'Autor = {Max Mustermann}', that is, the value is put
% into curly braces. You can use the \UzLThesisSetup command
% repeatedly and the order in which you provide the keys is not
% important. 
%
% Everything shown on the title page must be in German -- even
% if the thesis is written in English! Just insert German text for
% German keys and English text for English keys (like 'Abstract' needs
% English text, while 'Zusammenfassung' needs German text).

\UzLThesisSetup{
  %
  % !!!!!!!!!!!!!!!!!!!!!!!!!!!!!!!!!!
  % !!! Your action is needed here !!!
  % !!!!!!!!!!!!!!!!!!!!!!!!!!!!!!!!!!
  %
  % First, specify the institut or clinic at which the thesis was
  % written. You get the logo file from them (make sure it has the
  % correct size, namely the same as the example). If they do not have
  % a logo, the university's default logo is used.
  %
  % The 'verfasst' gets two arguments. Change the first to {an der}
  % for clinics, as in 'Verfasst = {an der}{Medizinischen Klinik I}'
  %
  Logo-Dateiname        = {uzl-thesis-logo-itcs.pdf},
  Verfasst              = {am}{Institut für Theoretische Informatik},
  %
  % The titles:
  %
  Titel auf Deutsch     = {
     Sichere Steganographie auf ML-Basierten Kanälen
  }, 
  Titel auf Englisch    = {
    Secure Steganography on ML-Based Channels
  },
  %
  % Author and supervisor:
  % 
  % Note that the 'Betreuer' or 'Betreuerin' is the supervisor, that
  % is, the professor who officially supervises the thesis. If there
  % is also an assistent of the professor who helped (typically a
  % lot), use 'Mit Unterstützung von' to thank that person. If the
  % thesis was mainly written 'externally' at some company or another
  % institute, point this out using 'Weitere Unterstützung'. 
  % 
  % For your own name, do *not* add things like "BSc" or "BSc
  % cand.". For the supervisor, you should normally include
  % "Prof. Dr." or "PD Dr." (ask your supervisor, what is
  % appropriate), but nothing more (so no
  % "Univ.-Prof. Dr. Dr. h.c. mult." unless your supervisor insists).  
  %
  Autor                 = {Jeremy Boy},
  Betreuer              = {Prof. Dr. Maciej Liśkiewicz},
  % 
  % Optional: Supporting persons and institutions. The text should be
  % in German, even for an English thesis.
  %
%  Mit Unterstützung von = {Harry Hilfreich},
  % 
  %   Weitere Unterstützung = {
  %     Die Arbeit ist im Rahmen einer Tätigkeit bei der Firma Muster GmbH
  %     entstanden.
  %   },
  %
  %
  % Your Degree Programm (Studiengang)
  %
  % Specify 'Bachelorarbeit' or 'Masterarbeit' and the degree
  % programme. Make sure the name of programme is correct and not
  % some abbreviation or some incorrect variant. For instance:
  % 'Medizinische Ingenierwissenschaft', but not 'MIW';
  % 'Medizinische Informatik', but not 'Medizin-Informatik';
  % 'Informatik', but not 'Informatik (SSE)'.
  %
  % Use German names for German programmes and English names for
  % English ones, so 'Infection Biology', not 'Infektionsbiologie'. 
  % For programmes that have a German bachelor and an English master,
  % use the German name for a bachelor thesis and the English name for
  % the master thesis.
  %
  Bachelorarbeit,
  Studiengang           = {IT-Sicherheit},
  %
  % Date on which the thesis is turned in German, formatted the
  % traditional German way:
  %
  Datum                 = {1. Oktober 2022},
  %
  % The English abstract. You must always provide abstracts in German
  % and in English. 
  %
  Abstract              = {
    Governmental and commercial adversaries increasingly threaten the confidentiality of encrypted communication.
    Steganography alllows confidential communication even in environments where cryptography is actively prohibited.
    The Meteor Stegosystem extends classical steganographic primitives for the use on channels based on generative models, such as GPT-2.
    Here, the secret text is embedded in the model's sampling scheme.
    In this thesis, I perform a statistical analysis of Meteor's performance and compare it with the results presented in the original paper.
    Subsequently, I present and evaluate different strategies for embedding secret text.
    The strategies optimize the length of generated messages against reliability and computational ressources required to generate and decrypt stegotexts.
    The strategies improve the reliability significantly while keeping achievable information density and required computational ressources justifiable.
  },
  Zusammenfassung       = {
    Staatliche und privatwirtschaftliche Akteure bedrohen zunehmend die Vertraulichkeit verschlüsselter Kommunikation.
    Steganographie ermöglicht vertrauliche Kommunikation auch dann, wenn die Nutzung klassischer Kryptographie verhindert wird.
    Das Meteor-Stegosystem erweitert klassische steganographischer Primitive für die Verwendung auf Kanälen, die auf generativen Modellen basieren, wie etwa GPT-2.
    Hierbei wird der Geheimtext in das Sampling des generativen Modells eingebettet.
    In dieser Thesis führe ich eine statistische Analyse der Performance des Meteor-Stegosystems durch und vergleiche diese mit den im Paper präsentierten Ergebnissen.
    Anschließend präsentiere und evaluiere ich unterschiedliche optimierte Strategien zur Einbettung des Geheimtexts.
    Die Strategien optimieren die Länge der erzeugten Nachrichten gegen Zuverlässigkeit und Berechnungsaufwand beim Erzeugen sowie Entschlüsseln von Stegotexten.
    Die Strategien verbessern die Zuverlässigkeit signifikant, während die erzielbare Informationsdichte und benötigter Rechenaufwand vertretbar bleiben.

  },
  %
  % Optional: 'Danksagungen' (German) or 'Acknowledgements'
  % (English). Both keys are optional and both have the same effect of
  % adding an acknowledgements text after the abstracts and before the
  % table of contents.
  %
  %Acknowledgements      = {
  %  This is the place where you can thank people and institutions, do
  %  not try to do this on the title page. The only exception is in
  %  case you wrote your thesis while working or staying at a company or abroad. Then you
  %  should use the \Latex{Weitere Unterstützung} key to provide a text
  %  (in German) that acknowledges the company or foreign
  %  institute. For instance, you could use texts like »Die Arbeit
  %    ist im Rahmen einer Tätigkeit bei der Firma Muster GmbH
  %    entstanden« or »Die Arbeit ist im Rahmen eines
  %    Forschungsaufenthalts beim Institut für Dieses und Jenes an der
  %    Universität Entenhausen entstanden«. Do not name and thank
  %    individual persons from the company or foreign institute on the
  %    title page, do that here. 
  %},
  % Bibliography style: Choose between
  % 
  % 'Alphabetische Bibliographie'
  % for all degree programmes in the natural sciences 
  % 
  % 'Numerische Bibliographie'
  % alternative for all other degree programmes
  % 
  % Either will load biblatex and setup the citation methods and the
  % bibliography styles correctly. You should not mess with them.
  % 
  Alphabetische Bibliographie,
  % Alternatively:
  % Numerische Bibliographie
}




%%%%%%%%%%%%%%%%%%%%
%
% Styling the thesis
%
%%%%%%%%%%%%%%%%%%%%
%
% Creating a visually pleasing layout and choosing fonts is not
% easy. Furthermore, different people have different preferences. Of
% course, for the University of Lübeck, the dean of studies could just
% force everyone to use one specific layout and font, but that seems a
% bit drastic and, also, it seems nice that thesis by different people
% have an individual style even though they all stick to the same
% overall structure.
%
% For these reasons, I (Till Tantau) have spend quite some time on
% designing a flexible layout and styling mechanism for theses.
%
% Basically, the overall structure of the thesis is fixed by the
% thesis class and so are many structural elements. For instance, you
% cannot change the order in which the abstract and table of contents
% are shown, you cannot move the bibliography elsewhere, indeed, the
% bibliography style is also fixed. Likewise, the text on the title
% page is fixed.
%
% Although many things are fixed, you *can* change several other
% things. For instance, you can change the font used for the main
% text, you can change which font is used for titles and headings or
% you can change whether titles and headlines are centered or flushed
% left.
%
% There are many LaTeX packages for changing such things. You are
% kindly asked *not to use them*. Rather, use (only) the options
% offered by the thesis class. All possible choices and combinations
% there have been tested by me and produce nice results; what happens
% with other packages no one knows and might no longer conform to what
% is expected by the university. As you will see, you still have a
% lot of options.
%
%
% Technical note: All styling is done via the command
%
% \UzLStyle{...}
%
% where ... is a key-value list just as for \UzLThesisSetup. The
% difference is just that everything having to do with styling as
% controlled by \UzLStyle, while the more “formal” setup keys are
% controlled by \UzLThesisSetup.
%
%%%
%
% Designs
%
%
% A \emph{design} is a whole set of font and layout options bundled
% together. They have been chosen in such a way that a visually
% pleasing “overall appearance” results.
%
%
% \UzLStyle{computer modern oldschool design}
%
% The look of this design mimics the “classical” way a paper or report
% created with \LaTeX\ looks like: The Computer Modern font is used,
% bold face fonts are used for headlines, only black and white are
% used as colors. This design reminds me of older scientific
% documents, especially from the computer science community where
% \LaTeX\ was used very early.
%
%
% \UzLStyle{computer modern basic design}
%
% A slightly less “oldschool” version of the previous design. It is
% still a classic design in the sense that it uses the Computer Modern
% font and that it still has this “good old \LaTeX” look, but some
% more modern aspects (like colors!) have been added.
%
% Note that this design uses Myriad for the title page (one of the
% “modern aspect”), which means that his font must be installed.
%
%
% \UzLStyle{computer modern scholary design}
%
% In my opinion, this is the ultimate “scholary design”: The thesis
% will look like it has been typeset by hand some 150 years ago and
% then printed by a university press. There is really nothing “modern”
% about it and the word in the name of the design is just part of the
% name of the “Computer Modern” font.
%
%
% \UzLStyle{pagella basic design}
%
% A, well, basic design that uses the Pagella font rather than the
% Computer Modern font. Especially the bold face version of this font
% looks nicer than the Computer Modern counterpart. Also, Pagella,
% while still having a “bookish” look, still feels a bit fresher than
% Computer Modern. 
%
%
% \UzLStyle{pagella centered design}
%
% A variant of the basic Pagella design that centers all
% headlines. A nice alternative to the basic version.
%
%
% \UzLStyle{pagella contrast design}
%
% This design tries to create some visual friction by contrasting the
% sans serif headline font (in bold!) with the main text. I find it a
% visually very interesting combination.
%
%
% \UzLStyle{alegrya basic design}
%
% The third variant of the basic design, this time using the Alegrya
% font. 
%
%
% \UzLStyle{alegrya scholary design}
%
% The Alegrya version of the previous “scholary” design. Unlike the
% Computer Modern version, this design does not look old, but more
% fresh -- while still creating the impression that the text must be
% about a very scientific subject. 
%
%
% \UzLStyle{alegrya stylish design}
%
% The design is quite similar to the scholary version for the Alegrya
% font, but with even more modern additions. “Stylish” is the word
% that comes to my mind.
%
%
\UzLStyle{alegrya modern design}
%
% A design that uses the sans serif version of the Alegrya font for
% the headlines. This is a nice modern overall design.
%
%%%




%%%%%%%%
%
% Now, include the package you need here using \usepackage. 
%
% However, many standard packages are already loaded by the class:
%
% amsmath, amssymb, amsthm, babel, biblatex, csquotes, etoolbox,
% filecontents, fontspec, geometry, hyperref, tikz (with libraries
% arrows.meta, positioning and shapes), varioref, url 
%
% Indeed, in many cases you will not need any extra packages.
%
%%%%%%%

\usepackage[inline]{enumitem}





\begin{document}

%
% The title page and table of contents will be inserted automatically
% here. 
%

% In a German thesis write: \chapter{Einleitung}


% !!!!!!!!!!!!!!!!!!!!!!!!!!!!!!!!!!
% !!! Your action is needed here !!!
% !!!!!!!!!!!!!!!!!!!!!!!!!!!!!!!!!!
%
% Replace with your own introduction:

\chapter{Introduction}

When governmental and private-sector adversaries try to breach the confidentiality of private communication, civil society requires means to defend their privacy.
Especially in authoritorian states, law enforcement continuously attempts to break or prohibit effective means of cryptography, especially when used for censorship-resistant communication.
\section{Steganography}

Steganography extends the concepts of cryptography to hide the mere existence of a message.
This makes it robust against censorship and prohibition.
While cryptography can be used to hide the contents of a confidental message, its use is easily discoverable by adversaries.
Steganography is usually modelled using the Prisoner's Problem, as introduced by \cite{Simmons83}.
There, two prisoners Alice and Bob are imprisoned.
While they can exchange messages, their contents are surveilled by a Warden.
Alice and Bob now want to craft and exchange escape plans.
Warden observes messages exchanged between Alice and Bob.
He tries to distinguish the exchange of the escape plan from innocous messages.

In steganography, a message is represented by a sequence of documents from an underlying distribution.
We call 
Most approaches to steganography use photographs as underlying distribution and hide information in the images' noise.
If the original photograph is unknown to Warden, Alice can encode hidden text in the original photograph

\section{Generative Neural Networks}

Generative Neural Networks (GNN) establish a model which approximate realistic distributions such as natural language or photographs.
In the last few years, the artificial intelligence community achieved tremendous progress in building powerful machine learning based models.
The Generative Pretrained Transformer 2 (GPT-2) by OpenAPI is one of the industry-standard approaches to (not only) text generation.
This model works on a set of tokens, which are words or sub-words of the approximated distribution.
To generate a sentence, i.e. a sequence of tokens, the model requires a history as input and outputs a probability distribution for the next token, see \ref{fig-generative-network}.
To generate the next token in the sequence, there are basically two approaches:
\begin{enumerate*}[label=(\roman*)] \item \label{enum-gnn-most-likely} selecting the most likely token from the distribution or \item \label{enum-gnn-sample} sampling according to the generated distribution \end{enumerate*}.
While \ref{enum-gnn-most-likely} creates determinstic output, the generated sequences tend to be repetitive.
On the other hand, approach \ref{enum-gnn-sample} is not deterministic, but usually creates higher quality output.
For a comparison of GNN model outputs, see examples \ref{ex-gpt2-output} and \ref{ex-gpt2-output-sample}.


\begin{example}[Example of GPT-2 model output without sampling]
	Hello, I'm a language model, not a programming language. I'm a language model. I'm a language model. I'm a language model
	\label{ex-gpt2-output}
\end{example}


\begin{example}[Example of GPT-2 model output with sampling]
	Hello, I'm a language model, I'm a problem solver in languages."
	At the same time, she said we can understand an
	\label{ex-gpt2-output-sample}
\end{example}


\begin{figure}[htpb]
	\centering
	\begin{tikzpicture}
            \node[rectangle, draw=black, minimum width=4cm, minimum height=4cm] (oracle) {ML Model};
            \node[left=of oracle] (history) {history};
            \node[above right=-5mm and 1cm of oracle] (t1) {$t_1$};
            \node[below=5mm of t1] (t2) {$t_2$};
            \node[below=5mm of t2] (t3) {$t_3$};
            \node[below=5mm of t3] (t4) {$\dots$};
            \node[below=5mm of t4] (tn) {$t_n$};
            \draw[->] (history) -- (oracle);
            \draw[->] (oracle) -- node[midway,below] {$p_1$} (t1);
            \draw[->] (oracle) -- node[midway,below] {$p_2$} (t2);
            \draw[->] (oracle) -- node[midway,below] {$p_3$} (t3);
            %\draw[->] (oracle) -- (t4);
            \draw[->] (oracle) -- node[midway,below] {$p_n$} (tn);
	\end{tikzpicture} \label{fig-generative-network}
	\caption{Generative ML Model as black-box function, taking a history as input.}
\end{figure}

\section{The Meteor Stegosystem}
\section{Ambiguous Tokenization}
\section{Structure Of This Thesis}
% In a German thesis write: \section{Aufbau dieser Arbeit}


% !!!!!!!!!!!!!!!!!!!!!!!!!!!!!!!!!!
% !!! Your action is needed here !!!
% !!!!!!!!!!!!!!!!!!!!!!!!!!!!!!!!!!
%
% Replace the following by one or two paragraphs describing the
% thesis's structure.
\chapter{Statistical Analysis Of The Meteor Stegosystem}
\chapter{Moving Forward: Improved Tokenization Strategies}
\chapter{Conclusion}

% In a German thesis write: \subsection{Zusammenfassung und Ausblick}


% !!!!!!!!!!!!!!!!!!!!!!!!!!!!!!!!!!
% !!! Your action is needed here !!!
% !!!!!!!!!!!!!!!!!!!!!!!!!!!!!!!!!!
%
% Replace the following with your conclusion

TODO write a conclusion.

% Normally, the bibliography comes next at this point. Do *not* (try
% to) include further indices and tables like an index or
% a list of figures or a list of tables or such things. Nobody
% actually uses them and they just use up space. 
%
% You *can* however include a glossary, if this seems appropriate. It
% goes here as an unnumbered chapter. Most thesis will *not* need a
% glossary: a well-written text (re)explains strange words and
% concepts as necessary. However, there are situations where a
% glossary may be helpful.














%%%
% 
% Bibliographies
%
%%%
%
% The uzl-thesis class will load biblatex for the bibliography
% management. This is a powerful package, see its documentation for
% details. The styles will be setup correctly and automatically by
% choosing one of the two style keys as described earlier.
%
% In order for the bibliography to work, run latex in the following
% order (which is the standard order):
% 
% > lualatex thesis-example
% > bibtex thesis-example
%e > lualatex thesis-example
% 
% Add BibTeX files using \addbibresource or use the {bibtex entries}
% environment (see below).
%
%%%
%
% Although everyting is normally setup automatically, you can change
% the options passed to biblatex using the key 'biblatex';
% for instance,
%
%   \UzLThesisSetup{biblatex={firstinits=false}}
%
% will switch off shortened first names. Normally, you will not need
% this key in your preamble. 
% 
% Note that the bibtex program is used as the 'backend' of biblatex
% by default (rather than biber, which is the preferred program of
% biblatex). This means that you can (and must) run *bibtex* after you
% have run lualatex on your thesis. If you wish to use biber instead
% of bibtex, say 'biblatex={backend=biber}'. 
% 
%%%
%
% The following environment is optional. It allows you to keep the
% bibtex entries for your thesis right here in the thesis file. What
% happens is that each time this tex file is processed, the contents
% of the following environment gets written to the file
% \jobname-bibtex-entries.bib (this file gets overwritten each
% time). Independently, \addbibresource{\jobname-bibtex-entries.bib}
% is always called if the file \jobname-bibtex-entries.bib
% exists. 
%
% In result, you can edit and keep the bibliography's bibtex entries
% right here. If you change something here, run latex, then bibtex,
% then latex once more.
%
% If you would like to manage the bibtex entries in a separate file,
% remove the below environment, delete the \jobname-bibtex-entries.bib
% file and instead write
%
% \addbibresource{filename-of-your-bibtex-file.bib}
%
% in the preamble.
%
%%%


% !!!!!!!!!!!!!!!!!!!!!!!!!!!!!!!!!!
% !!! Your action is needed here !!!
% !!!!!!!!!!!!!!!!!!!!!!!!!!!!!!!!!!
%
% Replace following example entries with the ones of your thesis.

\begin{bibtex-entries}
@article{Simmons83,
   author={Simmons, Gustavus J.},
   title={The Prisoners' Problem And The Subliminal Channel},
   year=1983,
   pages={1-2}
}
\end{bibtex-entries}



% If you need to have an appendix (I advise against it), insert it
% here using, first, \appendix and then \chapter and then,
% possibly, \section. 
%
% \appendix
%
% \chapter{Technical Appendix}
%
% \section{Experimental Parameters} % possibly
%
% Again, I advise against using an appendix.


\end{document}

%  LocalWords:  LaTeX tex moretexcs Lübeck pdf uzl lualatex bibtex th
%  LocalWords:  TechReport Kernighan Lamport's Tantau's Tantau cls kZ
%  LocalWords:  Mustermann emacs oldschool pdflatex texmf utf biber
%  LocalWords:  biblatex Alphabetische Bibliographie Numerische VIIa
%  LocalWords:  varioref german Einleitung Beiträge dieser Arbeit xml
%  LocalWords:  Ergebnisse Verwandte Arbeiten Aufbau nucleotide VIIc
%  LocalWords:  ensembl amino phylogenetic Alexa Siri decrypt versa
%  LocalWords:  cryptographic pre nondeterministic deterministically
%  LocalWords:  Beutelspacher Untersuchungen zum genetischen sep llcc
%  LocalWords:  Beispiel tikz jpg png Alegrya Kasimir Malewitsch PGF
%  LocalWords:  Lamport Institut für Theoretische Informatik zu url
%  LocalWords:  Universität Springer DowneyF Downey Parameterized doi
%  LocalWords:  BibLaTeX Kime Philipp urldate Mittelbach hyperref Lua
%  LocalWords:  Rahtz Oberdiek Heiko Braams Bezos López fontspec Das
%  LocalWords:  Arseneau amsmath ist Tipps und zur Formulierung
%  LocalWords:  mathematischer Gedanken Mathematik Studienanfänger
%  LocalWords:  Albrecht Vieweg Teubner Verlag
