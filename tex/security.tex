\chapter{Improving the Security of Meteor}
\label{chap:security}

When we communicate using a steganographic protocol, security is of upmost importance.
We probably communicate in an environment that is very hostile against its participants.
Therefore, we should satisfy the most rigorous security requirements we can.

In \autoref{def:sec-hopper}, we established the notion of steganographic security against chosen hiddentext attacks (SS-CHA).
In this chapter, we will show that the Meteor stegosystem is secure against SS-CHA adversaries with query complexity of one, i.e. adversaries that are allowed to send up to one query to the encoding oracle.
Afterwards, we will argue that Meteor is insecure against adversaries with query complexity greater than one.
In conclusion, we will introduce a modification to the Meteor stegosystem that improves its security against SS-CHA adversaries by replacing the one-time pad used in Meteor with a symmetric encryption scheme in counter mode (SES-CTR).

\section{Security Against Chosen Hiddentext Attackers}

In \autoref{def:sec-hopper}, we introduced Hopper's notion of steganographic security against chosen hiddentext attackers (SS-CHA).
In this chapter, we will show that the Meteor stegosystem is secure against SS-CHA adversaries with query complexity of one and insecure against SS-CHA adversaries with query complexity greater than one.


\begin{theorem}
  Meteor is secure against SS-CHA adversaries with query complexity of one.
\label{thm:meteor-secure}
\end{theorem}
\begin{proof}
  We show SS-CHA security by reduction to an attack against the underlying PRG.
  Let $\mathcal{A}$ be an SS-CHA attacker against the Meteor stegosystem, i.e. $\mathcal{A}$ can distinguish the output of $Encode$ from that of a random oracle $\mathcal{O}$ after one query.
  We will now show that $\mathcal{A}$ is also an attacker against the underlying PRG.
  For a given query $(m,h)$, the attacker gets the response $c = c_0 ||c_1 || \dots || c_{\tau}$ from $Encode$ or $\mathcal{O}$, depending on the challenger's random bit.
  The attacker can then use $Prefix$ and $LenPrefix$ (as Bob would during decoding) to recover the bit string $r$ used to sample $c$ from the underlying distribution.

  How is $r$ constructed if the challenger used $Encode$ to generate $c$?
  It is the exclusive-or of $\beta$ bits of $m$ and a pseudorandom value $mask$ from a PRG.
  Since $mask$ is pseudorandom, the exclusive-or of $mask$ and some other value is pseudorandom as well.
  To distinguish the output of $Encode$ from some oracle $\mathcal{O}$, the attacker has to distinguish $r$ from a randomly sampled token, i.e. the attacker has to distinguish $r$ from a truly random value $r'$.
  Therefore, $\mathcal{A}$ can be used to attack the PRG.
  Since we expect the underlying PRG to be secure, $\mathcal{A}$ cannot distinguish $r$ from $r'$ with non-negligible probability in polynomial time.
\end{proof}

But is this sufficient to satisfy Hopper's definition of steganographic security?
Unfortunately not.
In Hopper's $(t,q,l)$-SS-CHA game, an attacker may perform up to $q(k)$ queries.
The aforementioned security proof only considers security against one query.

\begin{theorem}
  Meteor is insecure against SS-CHA adversaries with query complexity of at least two.	
\end{theorem}
\begin{proof}
Let $\mathcal{A}$ be an SS-CHA adversary with at least two queries.
Let $\mathcal{A}$ repeat the same query $(m,h)$ twice.
While the oracle $\mathcal{O}$ will~ --~ with overwhelming probability~ --~ sample an entirely different covertext from the distribution,  $Encode$ will output the same stegotext twice.
After only two queries, $\mathcal{A}$ is able to distinguish Meteor's outputs from those of a random oracle with high certainty.

We can easily verify this using the Meteor online demo \cite{MeteorDemo2021}.
For example, with query
\begin{lstlisting}
m = "Hello world"
h = "Despite a long history of research and wide-spread applications to censorship resistant systems, practical steganographic systems capable of embedding messages into realistic communication distributions, like text, do not exist.\n\n"
\end{lstlisting}
the Meteor online demo initialized with key (just for this example, Meteor is deterministic for any key $k$)
\begin{lstlisting}
k = 0x010101010101010101010101010101010101
      010101010101010101010101010101010101
\end{lstlisting}
deterministically generates the stegotext
\begin{lstlisting}
c = "- The great strength of steganography is the ability to send messages without any significant communication layer of encryption, Pinnew"
\end{lstlisting}
while random sampling from the distribution would -- with overwhelming probability -- yield a different covertext for each invocation of the encoding oracle.
Therefore, $\mathcal{A}$ has significant advantage against the challenger after only two queries.
\end{proof}

In the following section, we will propose a modification to Meteor that improves the security against SS-CHA adversaries by randomizing the stegotext returned by $Encode$.

\section{Improving Meteor's Security Against Chosen Hiddentext Attackers}

To improve Meteor's security, we have to randomize the output of $Encode$, i.e. on each invocation with the same inputs, $Encode$ should, with overwhelming probability, return a different stegotext while retaining indistinguishability from random sampling.

To achieve that we introduce a new cryptographic primitive, symmetric encryption schemes in counter mode with a pseudorandom function $F$ (SES-CTR with PRF).
We will introduce a definition of an SES which is equivalent to that introduced in \cite{Berndt2017}.
There, it is argued that SES satisfies the IND\$-CPA property if F is a secure pseudorandom function.

The idea behind encryption in SES-CTR is simple:
First, split the message $m$ into $n$ blocks of size $F.in(\lambda)$ specific to $F$.
Choose an $IV$ of size $F.in(\lambda)$ at random.
It is important to not reuse an $IV$ with a given key $k$ to retain One-Time-Pad security.
Now, for each block $m_i$, use $F.Eval$ to generate a pseudorandom value $r = F.Eval(k, IV+i)$.
Afterwards, calculate the exclusive or of $r$ and $m_i$ to get the next block $c_i$.
The ciphertext of size $(n+1)\cdot F.in(\lambda)$ is the $IV$ concatenated with ciphertext blocks $c_i$.

Decryption is similar to encryption.
First, split $c$ into $n+1$ blocks of size $F.in(\lambda)$.
Interpret the first block of $c$ as $IV$.
For each block $c_i$, calculate $r = F.Eval(k, IV+i)$.
Then, calculate the exclusive or of $r$ and $c_i$ to recover the message block $m_i = c_i \oplus r$.

Now that we introduced SES, we still need to find $F$ which is a PRF.
In practice, we can use AES as $F$, which is believed to be a pseudorandom permutation (PRP), with key $k$ and messages $IV+i$.

\begin{Pseudocode}[caption={SES Encrypt algorithm}]
algorithm $SES.Encrypt_F(k, m=m_1 m_2 \dots m_n \in \{0,1\}^{n\cdot F.in(\lambda)})$
Output: Ciphertext $c$
  $\lambda = |k|$
  $IV \leftarrowS \{0,1\}^{F.in(\lambda)}$
  for $i \in \{ 1, 2, \dots, n \}$ do
    $c_i \leftarrow F.Eval(k, (IV+i) \mod 2^{F.in(\lambda)}) \oplus m_i$
  Output $IV||c_1||c_2||\dots||c_n$
\end{Pseudocode}%
\begin{Pseudocode}[caption={SES Decrypt algorithm}]
algorithm $SES.Decrypt_F(k, c=c_0 c_1 \dots c_n \in \{0,1\}^{(n+1)\cdot F.in(\lambda)})$
  Output: cleartext $m$
  $\lambda = |k|$
  $IV \leftarrow c_0$
  for $i \in \{ 1, 2, \dots, n \}$ do
    $m_i \leftarrow F.Eval(k, (IV+i) \mod 2^{F.in(\lambda)}) \oplus c_i$
  Output $m_1||m_2||\dots||m_n$
\end{Pseudocode}%

In \autoref{alg:encode-nd}, we use this SES to encrypt the hiddentext message.
Since $IV$ as well as all blocks $c_i$ are indistinguishable from randomness, the entire output of $SES.Encrypt_F$ is indistinguishable from randomness.
In \autoref{alg:decode-nd}, the SES is used to decrypt the hiddentext.
Since $SES.Encrypt$ yields different ciphertexts on each invocation (even with the same message $m$ and key $k$ because $IV$ is sampled at random on each invocation), an attacker can only distinguish the value $r \leftarrow SES.Encrypt(k,m)$ used to sample a stegotext from randomness if the same $IV$ is generated twice for the same pair $(m,k)$, which is very unlikely.

These modifications improve Meteor's security against SS-CHA adversaries.
Repeated invocations of $EncodeRandomized$ return different stegotexts with overwhelming probability.
Since $SES$ is IND\$-CPA secure we can, with the same reduction as in Theorem \ref{thm:meteor-secure}, show SS-CHA security against polynomial attackers as long as the same $IV$ is not generated twice, which is -- in practice -- very unlikely for sufficient $IV$ lengths.


\begin{Pseudocode}[float, caption={
$EncodeRandomized$ algorithm.
This algorithm differs from the original $Encode$ algorithm by using $SES$ to encrypt the message.
The ciphertext (which is indistinguishable from randomness and indeterministic) is then used to sample from $\mathcal{M}$.
}, label={alg:encode-nd}]
algorithm $EncodeRandomized_{\mathcal{M}}^{\beta}(k_{prg}, m, h)$
  Output: Stegotext message $c$
  $c \leftarrow \epsilon,~ i \leftarrow 0,~ n \leftarrow 0$
  $r \leftarrow SES.Encrypt(k, m)$
  while $n < |r|$ do
    $c_i \leftarrow Sample_{\mathcal{M}}^\beta(h, r[n: n+\beta])$
    $\mathcal{R} \leftarrow Recover_{\mathcal{M}}^\beta(h, c_i)$
    $n_i \leftarrow LenPrefix^\beta(\mathcal{R})$
    $c \leftarrow c || c_i,~ n \leftarrow n+n_i,~ h \leftarrow h||c_i,~ i \leftarrow i + 1$
  Output $c$
\end{Pseudocode}
\begin{Pseudocode}[float, caption={
$DecodeRandomized$ algorithm.
This algorithm differs from the original $Decode$ algorithm by using $SES$ to decrypt the stegotext.
}, label={alg:decode-nd}]
algorithm $DecodeRandomized_{\mathcal{M}}^{\beta}(k_{prg}, c, h)$
  Output: Plaintext message $m$
  $m \leftarrow \epsilon,~ n \leftarrow 0,~ r \leftarrow \epsilon$
  Parse $c$ as $c_0 || c_1 || \dots || c_{\tau}$
  for $i \in \{ 0, 1, \dots, \tau \}$ do
    $\mathcal{R} \leftarrow Recover_{\mathcal{M}}^\beta(h, c_i)$
    $r_i \leftarrow Prefix^\beta(\mathcal{R})$
    $r \leftarrow r || r_i$
    $h \leftarrow h||c_i$
  $m \leftarrow SES.Decrypt(k, r)$
  Output $m$
\end{Pseudocode}