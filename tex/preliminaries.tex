
\chapter{Preliminaries}
\label{chap:preliminaries}
In this chapter, we will introduce notation that will be used in this thesis.

We denote $x \leftarrowS S$ as a value $x$ sampled uniformly at random from the set $S$.
When working with a probability distribution $D$, $x \leftarrowS D$ denotes the random sampling of a value according to $D$.
We also denote the probability of some value $x \in D$ as $Pr_D[x]$.

We define $U_n$ as the uniform distribution over bitstrings of length $n$.

In code samples, we denote $x \leftarrow a$ as assignment of value $a$ to variable $x$.
When working with sorted sequences such as arrays or strings we use pythonic slicing notation.
Given a sequence $x = (x_0, x_1, \dots, x_{n-1})$ of length $n$, we use the following slicing operations:

\begin{itemize}
  \item $x[i] = x_i$
  \item $x[i:j] = (x_i, x_{i+1}, \dots, x_{j-1})$
  \item $x[i:] = x[i:n] = (x_i, x_{i+1}, \dots, x_{n-1})$
\end{itemize}

We will use $x \leftarrow \epsilon$ as the empty value assignment.
We also use the infix operator $\oplus \colon \{0,1\}^n \times \{0,1\}^n \rightarrow \{0,1\}^n,~ a \oplus b$ for the exclusive or of values $a$ and $b$.
The (infix) concatenation operator $|| \colon X^n \times X^m \rightarrow X^{n+m},~ a||b$ appends the value or sequence $b$ to the value or sequence $a$. 

The variable $\lambda$ denotes the security parameter used in cryptographic primitives.
The set $\mathcal{K}$ denotes the set of keys of a cryptographic primitive, the variable $k \in \mathcal{K}$ is a key.
With $\mu(\lambda)$, we denote a function negligible in $\lambda$.

With steganographic protocols, we will use the variable $m \in \{0,1\}^*$ as hiddentext messages and set of histories $\mathcal{H}$ with specific history $h \in \mathcal{H}$.
