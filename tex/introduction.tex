\chapter{Introduction}

When governmental and private-sector adversaries try to breach the confidentiality of private communication, civil society requires means to defend their privacy.
Especially in authoritorian states, law enforcement continuously attempts to break or prohibit effective means of cryptography, especially when used for censorship-resistant communication.

\section{Motivation}

TODO

\section{Structure Of This Thesis}

Chapter 2 discusses previous works.
First, we will discuss steganography and provably secure steganography, introduced in \cite{Hopper04}.
Afterwards, we discuss how generative neural networks (GNNs) such as GPT-2 produce high-quality text iteratively.
Concluding, I will present the Meteor stegosystem, a provably secure steganographic system over ML-based channels as introduced by XYZ.

Chapter 3 provides an analysis of the correctness of the Meteor stegosystem. 
We will see that the Meteor stegosystem is not correct because of the way tokenization in GNNs work.
Afterwards, I will present algorithms which ensure correctness of Meteor on the receiver site.
Unfortunately, these algorithms pose overhead exponential in the length of the generated stegotext.

\todo{calculate introduced overhead}

In chapter 4, I present a small modification to the Meteor stegosystem which allows two-way communication.
There, we embed a fixed amount of bits in each message, using the DialoGPT model developed by Microsoft \cite{Zhang2020} to generate dialog-like messages.
This modification makes sure that the algorithms from chapter 3 can be applied more conveniently, since the exponential overhead is negligible for shorter messages.

In chapter 5, we will perform a security analysis of the Meteor stegosystem using Hopper's definition of secure stegosystem.
We will compare the definition of security used by the Meteor authors with the definition of security by Hopper.
We then will show that the Meteor stegosystem is not universally secure in Hopper's definition.

Finally, in chapter 6, I will discuss the results in a conclusion and give ideas for further research.