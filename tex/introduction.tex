\chapter{Introduction}

\section{Motivation}

When governmental and private-sector adversaries try to breach the confidentiality of private communication, society requires means to defend their privacy.
Especially in authoritorian states, law enforcement continuously attempts to break or prohibit effective means of cryptography, especially when used for censorship-resistant communication such as Tor.
Steganography extends the concepts of cryptography by not only hiding the contents of a message, but also hiding the mere existence of a hidden message.
If we are able to establish provably secure steganographic protocols -- or stegosystems -- we can avoid those 


\section{Structure Of This Thesis}

\autoref{chap:previous-work} discusses previous works.
First, we will discuss steganography and provably secure steganography, introduced in \cite{Hopper04}.
Afterwards, we discuss how generative neural networks (GNNs) such as GPT-2 produce high-quality text iteratively.
Concluding, I will present the Meteor stegosystem, a provably secure steganographic system over ML-based channels as introduced by \cite{Meteor2021}.

\autoref{chap:correctness} provides an analysis of the correctness of the Meteor stegosystem. 
This chapter relies on the material presented in \autoref{chap:previous-work}, especially the definitions of steganographic correctness established in \autoref{def:correctness-hopper} and \autoref{def:correctness-kaptchuk} as well as the discussion of generative neural networks in \autoref{sec:generative-neural-networks}.
We will see that the Meteor stegosystem is not correct because of the way tokenization in GNNs work.
Afterwards, I will present algorithms which ensure correctness of Meteor on the receiver site.
Unfortunately, these algorithms pose overhead exponential in the length of the generated stegotext.

In \autoref{chap:twowaycommunication}, I present a small modification to the Meteor stegosystem which allows two-way communication.
This chapter relies on the material presented in \autoref{sec:meteor} and \autoref{sec:generative-neural-networks}.
There, we embed a fixed amount of bits in each message, using the DialoGPT model developed by Microsoft \cite{Zhang2020} to generate dialog-like messages.
This modification allows users of the Meteor stegosystem to communicate more naturally in the context of instant messaging.
It also makes sure that the algorithms from \autoref{chap:correctness} can be applied more conveniently, since the exponential overhead is still manageable for shorter messages.

In \autoref{chap:security}, we will perform a security analysis of the Meteor stegosystem using Hopper's definition of secure stegosystem.
This chapter relies heavily on the material presented in \autoref{chap:previous-work}, especially the security definitions in \autoref{def:sec-kaptchuk} and \autoref{def:sec-hopper}.
We will compare the definition of security used by the Meteor authors with the definition of security by Hopper.
We then will show that the Meteor stegosystem is not universally secure in Hopper's definition.
Concluding, I will present a modification to the Meteor stegosystem which increases its security against chosen hiddentext attackers.

Finally, in \autoref{chap:conclusion}, I will discuss the results in a conclusion and give ideas for further research.